\chapter{Background \& Objectives}\label{ch:background-&-objectives}


\section{Introduction}\label{sec:introduction}

In the world of Satellite Imagery analysis, the tools needed to provide robust and efficient classification and unlock the
full potential of satellite imagery across various disciplines such as environmental monitoring, disaster response and
urban planning is crucial.
This project was initiated by recognizing this need for an application that leverages machine learning to simplify and enhance the classification of
satellite imagery

The Selection of the K-means clustering algorithm for this task was for several reasons.
Its simplicity, effectiveness, and arguably the most important that it is unsupervised as opposed to the CAST (Decision trees),
Random forrest and Artificial Neural Network approaches which are supervised learning methods.
The K-means classifier allows the segmentation of large datasets of satellite imagery into meaningful clusters based on Pixel similarity.
This approach helps in categorizing Land use and Specifically in detecting changes over time in regard to land use as we will see.

\section{Background}\label{sec:background}

The main idea of this project centers on the advanced utilization of satellite imagery for practical real world issues\cite{urbanEcology2022}.
Satellite imagery provides a view of the earth's surface that most individuals don't usually look at, it provides critical data for environmental monitoring,
urban development, strategic planning, land use control, security issues, and many more useful and important applications.
The application of machine learning algorithms to satellite imagery enhances the ability to categorize and analyze the data more efficiently
and in a less time-consuming way.

\subsection{Problem Overview}\label{subsec:problem-overview}

The increase in satellite imagery presents both an opportunity and a challenge.
With satellites continuously orbiting and capturing visuals of earth's surface, there exists a vast collection of images
that contain data which can enhance the decision taken in certain fields across multiple sectors.
However, the volume of this data makes it difficult to process and analyze efficiently with conventional methods.
Which may be inaccurate and or incredibly slow and can be influenced to ignore certain information based on biases and or pressure
to complete the work on time.
This can cause issues for time-sensitive matters such as environmental changes, disaster response and or Urgent urban developmental needs.

Thus, the problem requires advanced tools that can handle the large amount of data taken from satellite imagery and convert it
into usable insights swiftly and accurately.
Traditional Image processing methods often fall short in managing the spatial and spectral diversity found in satellite imagery.
Other Machine Learning Methods, specifically those most used for land use classification, require to be trained on data which takes time to collect and
is expensive to store depending on how much data is collected.
Not to mention, the training itself will take time that you might not have in an Emergency situation\cite{ahadflooddtec2022}, and the model will be only as good as the training data
used to train it.

Because of these Issues, The development of an application that uses unsupervised algorithms for satellite imagery classification
addresses these challenges.
By using an unsupervised approach to classification, it saves time on training and allows the user to segment the image by clusters which represent
different land uses and makes it easier to see difference over time in land use in the area.
It aims to enhance the accessibility and utility of satellite imagery.
making it actionable for experts and informative for the layman.

\subsection{Problem analysis}\label{subsec:problem-analysis}

To address the issues of satellite imagery analysis, A solution must include several key features.
these features would aim to ensure that the system not only meets the immediate needs of various users but also
adapt to evolving technological changes.
These include the following:

\begin{itemize}
    \item \textbf{Real-Time Processing:}
    \textit{ The ability to process data and analyze it swiftly is crucial. Developing systems that offer near realtime processing capabilities ensure actionable insights when }
    \item \textbf{sufficient in time and cost }
    \textit{Should be quick and easily used on standard computer hardware. It should minimize setup times and make
    efficient use of hardware resources to ensure cost-effectiveness and accessibility to a broader range of users}
    \item \textbf{Accessibility and Usability:}
    \textit{Should be able to be used and be understood by experts and laymen. this means a Simple-to-use user interface with
    simple to understand Image Analysis that is actionable to the users specific needs}
    \item \textbf{Accuracy:}
    \textit{The program needs to be accurate to a certain degree to be actionable.To achieve it needs to be able
    to give metrics on its performance after completion so that the user can make informed decisions about the results}
    \item \textbf{Difference Map:}
    \textit{The program should be able to compare the area to a previous version of the area and to tell the user visually how much change has occurred}
    \item \textbf{Unsupervised Machine learning Method:}
    \textit{The program should use an unsupervised machine learning method}
    \item \textbf{Continuous Improvement and Learning:}
    \textit{Since this field is ever evolving the system must be designed for easy maintenance and updates. The code should be clear and well-documented,
        facilitating the integration of new features and the adaptation of the system to incorporate the latest research findings and technological improvements.}
\end{itemize}

\section{ Project Analysis}\label{sec:project-analysis}

This project aims to develop a sophisticated tool for analyzing satellite imagery, leveraging the capabilities of unsupervised machine learning technologies.
Opting for an unsupervised machine learning approach provides significant advantages over supervised methods,
particularly in scenarios where labeled data is scarce or costly to obtain.

Supervised methods require extensive labeled datasets for training, which can be both time-consuming and expensive to prepare,
especially for complex image datasets like satellite imagery.
In contrast, unsupervised machine learning automates the analysis without the need for labeled data,
significantly enhancing speed and reducing costs.
By integrating advanced unsupervised algorithms, the system can process vast amounts of imagery data,
identifying inherent patterns and changes more quickly and accurately than supervised methods.

This automation not only accelerates the analytical process but also reduces reliance on potentially biased or inconsistent manual labeling.
Furthermore, unsupervised systems can continuously adapt and improve as they are exposed to new data,
learning from the inherent structure and variations without the need for continuous retraining with new labeled data.

Moreover, leveraging unsupervised machine learning reduces the overhead associated with maintaining and updating large labeled datasets,
offering a unified approach that can be easily scaled and adapted to various needs without significant modifications.
This adaptability is crucial for dynamic applications like environmental monitoring and urban planning\cite{dynamicMonitoring2022}, where conditions can
evolve rapidly and unpredictably.

Ultimately, this scalability and cost-effectiveness make advanced satellite image analysis more accessible to a broader range
of users, from experts to laymen in related fields, enhancing the potential impact of the technology across multiple sectors.

\subsection{Supervised vs. Unsupervised}\label{subsec:supervised-vs.-unsupervised}

As stated in the previous few sections, an unsupervised approach would be better than a supervised approach for this project
here is why:

\begin{itemize}
    \item \textbf{Elimination of Labeling Requirements:}
    \textit{ One of the most significant challenges with supervised learning is the necessity for labeled datasets, which are used to train the models.
    Labeling satellite images can be labor-intensive and costly, requiring expert knowledge and a considerable amount of time.
    An unsupervised approach, however, does not require labeled data.
    It automatically detects patterns and anomalies in the data based on the inherent characteristics of the images, such as pixel values and textures.}
    \item \textbf{Adaptability to New Data:}
    \textit{Satellite imagery is continuously evolving due to changes on the Earth's surface and differences in data acquisition techniques.
    Supervised models often need retraining or fine-tuning when new types of data are introduced, which can be a cumbersome and resource-intensive process.
    Unsupervised learning methods are inherently more flexible, adapting to new patterns and changes in the data without the need for retraining.
    This adaptability makes unsupervised learning particularly suitable for the dynamic nature of satellite imagery, }
    \item \textbf{Scalability and Cost-Effectiveness:}
    \textit{Handling the vast amounts of data generated by satellite technologies can be more scalable under an unsupervised framework.
    Since there is no need to label new data, continually}
    \item \textbf{Discovery of Unknown Patterns, }
    \textit{Unsupervised learning is particularly adept at identifying patterns and clusters that were not previously anticipated.
    In satellite imagery, this capability is crucial as it can reveal unexpected changes or new phenomena that a supervised model,
        trained on previously known labels, might overlook.}
    \item \textbf{Reduced Bias and Greater Objectivity:}
    \textit{Since Unsupervised Learning does not rely on Labels it is not susceptible to biases that might appear in the training dataset of a Supervised learning method}
\end{itemize}

All of these factors played a role in why the Project uses an unsupervised machine learning algorithm, specifically the previously mentioned K-means clustering
algorithm.
As it not only optimizes resource utilization but also enhances the analytical capabilities of the system.

\subsection{Choice of Machine learning Algorithm}\label{subsec:choice-of-machine-learning-algorithm}

In selecting The appropriate machine learning algorithm for the classification of satellite imagery in our project,
the K-means clustering algorithm from the sklearn library was chosen due to its specific characteristics that align well with the
requirements of the project, Specifically:

\begin{itemize}
    \item \textbf{Simplicity:}
    \textit{K-means is one of the simplest clustering algorithms to implement and understand. it's computationally efficient as well
    especially when dealing with large datasets, like those typically generated by satallite imagery.}
    \item \textbf{ Good metrics for calculating Accuracy of the Clusters:}
    \textit{Silhouette Score, Davies Bouldin Index, and Inertia provide valuable insights into the clustering performance, but they measure different aspects.
    The Silhouette Score assesses how appropriately data points have been grouped compared to other clusters -1 being it's in the
    wrong cluster and 1 being this is definitely the correct cluster, which reflects separation and cohesion,
        while Inertia focuses on the compactness of the clusters. A high number is not very compact meaning the cluster is not very uniform
        and a lower number meaning it's a compact cluster. Employing both metrics together allows for balancing cluster
        quality through Inertia and relevance through the Silhouette Score, facilitating more accurate and reliable cluster analysis.
        Additionally, the Davies-Bouldin Index (DBI) enhances this evaluation by measuring the ratio of intra-cluster distances to inter-cluster distances.
        A lower DBI value indicates better clustering as it signifies that the clusters are both compact and well-separated from each other.
        Integrating DBI into the assessment provides a comprehensive view of cluster validity, For the Specific image.}
    \item \textbf{Well-defined Clusters}
    \textit{K-means works best with well seperated data and can produce very tight clusters which are very useful in our project when we need to delineate
    different types of land use from each other}
    \item \textbf{Robustness and consistency:}
    \textit{K-means often produces consistent results, making it reliable for repeat analyzes under similar conditions.
    This predictability is valuable in applications requiring repeated deployment, like monitoring changes in environmental landscapes over time through satellite images.}
\end{itemize}



\subsection{Primary Objective}\label{subsec:primary-objective}

The primary objective of the project is to develop an advanced application capable of efficiently processing and
analyzing satellite imagery using a machine learning algorithm.
This tool aims to transform satellite data into actionable insights swiftly and accurately, enhancing decision-making
in various sectors.
By utilizing machine learning techniques the application seeks to outperform non-machine learning techniques by offering greater precision and faster
processing times, thereby providing rapid and cost-effective image analysis.
This will make satellite imagery more accessible and useful to both experts and laymen, helping them to understand
and react to changes in the landscape efficiently and effectively for the users specific uses.



\section{Process}\label{sec:process}

\subsection{Methodology}\label{subsec:methodology}

In the course of this project, Multiple Methodologies were considered, particularly agile methodologies.
Like Scrum, Extreme Programming, and Kanban, as well as plan driven ones like the Waterfall approach.
In the beginning of the project, the idea was to go for a plan driven Waterfall method.

Overtime however, FDD or Feature Driven Development was fully chosen as the methodology for its specific advantages in
managing software development projects through iterative and incremental progress, Unlike the Waterfall method which for a solo
developer puts too much strain in the beginning on rigid planning, making problems later on when designs or features have
to change for unspecified reasons.

\subsection{Feature Driven Development (FDD) Overview}\label{subsec:feature-driven-development-(fdd)-overview}

Feature Driven Development is a client-centric, architecture driven and pragmatic software development Methodology.
Its main focus is to deliver working software in a timely manner consistently.
FDD is usually used in a collaborative setting.
However, even as this was a solo project, it was straightforward to adapt it to being a
 single developer scenario unlike waterfall which is much less able to be adapted to a solo methodology

FDD compartmentalizes the planning into Features/functions of a whole overall model of the program which serves as a blueprint of the program.
Each feature is then given a priority in the plan, and in each iteration of the Program you plan design and code that feature and then move on to the next.
and by the time you finish, you have a fully implemented completely working program ready for the user to use and enjoy.

\subsection{Development environment, Libraries, and Programing language}\label{subsec:development-enviroment-libraries-and-programing-language}

For this project, it was fully developed in the anaconda development environment.
As it ensures that the program will be Cross platform, has a wide array of machine learning and non-machine learning libraries
which were essential for this project such as

\begin{itemize}
    \item \textbf{numpy: 1.23.5}
    \item \textbf{matplotlib: 3.80}
    \item \textbf{scikit-Learn: 1.30}
    \item \textbf{cv2: 4.60}
    \item \textbf{Pillow: 10.2.0}
    \item \textbf{tk: 8.6.12}
    \item \textbf{math}
\end{itemize}

It was also used as it supports python 3.11.5, which is the version of python this program was developed on.
Plus, there was already prior experience for this environment as it was used for a previous python module.
Although a new Fresh anaconda environment was created for this project instead of the old one being reused.

\subsection{Version Control System}\label{subsec:version-control-system}

For this project, GitHub was used as the version control system.
As GitHub is widely recognized for its robust functionality in managing changes to code, collaborating on a project, storing code securely
 over time, Documentation support, and Integration with other tools.
Such tools as The IDE Pycharm on which this program was developed

\subsection{Documentation of diagrams}\label{subsec:documentation-of-diagrams}
The simple UML diagrams used in this report have been created using plantUML scripts from  \cite{planttext.com} and the line graphs
were generated using code










