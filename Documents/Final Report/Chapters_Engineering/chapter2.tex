\chapter{Experimentation}\label{ch:experimentation}


\section{K-Means clustering}\label{sec:k-means-clustering}

This section is about how the Kmeans clustering algorithm works mathematically.
As it is important that we learn how the Classifier functions so that we can understand how Clustering occurs,
what the processes are, and how we can manipulate it to get the best possible results.
The steps are as follows:

\begin{itemize}
    \item \textbf{Initialization:}
    \textit{The first step in k means clustering involves selecting K initial centroids, where k is the number of clusters
    you want to identify in the dataset.
    The centroids can be randomly selected from the datapoints, or you can provide a heuristic or specific strategy
    on which they are chosen to enhance the quality and or speed of convergence.}
    \item \textbf{ Assignment step}
    \textit{In this step each data point in the dataset is assigned to the nearest centroid.
    the nearest is determined by the euclidian distance between the data point and the centroid this is mathematically determined by
    \[
        C_i = \{ x_p : \|x_p - m_i\| \leq \|x_p - m_j\| \, \forall j, 1 \leq j \leq k \}
    \]}
    \textit{where $C_i$ is the set of data points assigned to cluster $i$, $x_p$ is a data point, and $m_i$ is the centroid
    of cluster $i$. This means each data point $x_p$ is assigned to cluster $i$ if the distance from $x_p$ to $m_i$ is the smallest
    among all centroids.}
    \item \textbf{Update Step}
    \textit{Once all data points have been assigned to clusters, the centroids need to be recalculated.
    This is done by taking the mean of all points assigned to each cluster—the position of the centroid of each cluster is updated
    to the mean position of all points belonging to that cluster. The formula for updating the centroid of each cluster is:
    \[
        m_i = \frac{1}{|C_i|} \sum_{x \in C_i} x
    \]}
    \textit{where $|C_i|$ is the number of data points in cluster $i$, and $\sum_{x \in C_i}$ is the sum of all data points
    in cluste $i$}
    \item \textbf{Iteration}
    \textit{ In this step steps 2 and 3 are repeqted iterativly until the centroilds stop moving significantly, this means
    that the cluster is now stabilized and the algorithm has converged, the number of iterations has been spent. This allows
    the cluster assignmenst and centeroid positions to reflect the data accurately.}
    \item \textbf{Convergence}
    \textit{The algorithm has converged when the centroids have stabilized and or alternatively when the assignment of points to clusters
    longer change }
\end{itemize}

\section{Metrics Used}\label{sec:metrics-used}

When Employing Clustering algorithms such as K-means, assessing the quality of the algorithm is important as it validates
the effectiveness of the analysis.
Because of this, I use three crucial metrics to evaluate the K-means algorithm Silhouette score, Inertia and DBI(Davies-Bouldin Index)

\subsection{Silhouette Score}\label{subsec:silhouette-score}

The Silhouette Score is a measure of how similar an object is to its own cluster compared to other clusters.
The score is calculated for each data point and can range from -1 to +1, where a high value indicates that the object is
well-matched to its own cluster and poorly matched to neighboring clusters
Mathematically its definition is:
\[
    s(i) = \frac{b(i) - a(i)}{\max(a(i), b(i))}
\]
    $a(i)$ is the mean distance between $i$ and all other data points in the same cluster.

    This measures how well it $i$ is assigned to its cluster (the smaller, the better).

    $b(i)$ is the minimum mean distance from ii to all points in any other cluster, of which $i$ is not a member.

    This measures how poorly $i$ is matched to its neighboring cluster (the larger, the better).

The overall Silhouette Score for the dataset is the mean Silhouette Score of all individual points.
    This score provides a succinct measurement of how appropriately the data has been clustered.

\subsection{Inertia}\label{subsec:inertia}

Inertia, also known as the within-cluster sum of squares, measures the compactness of the clusters, which ideally should be as small as possible .
It is calculated by summing the squared distances between each data point and its nearest centroid.
Mathematically its definition is:
\[
   W(C) = \sum_{i=1}^k \sum_{x \in C_i} \|x - \mu_i\|^2
\]



$C_i$ is the set of all points in cluster $i$,

$mu_i$ is the centeroid of cluster $i$

Outer Sum: $\sum{i = 1}^k$ iterates over each cluster from 1 to $k$.

Inner Sum: $\sum{x \in C_i}$ sums over all points $x$ within each cluster $C_i$.

$ \|x - \mu_i\|^2$ computes the squared Euclidean distance between a point $x$ and the cluster centroid $mu_i$,
which is the norm squared of the vector difference.

$k$ is the number of clusters

\subsection{Davies-Bouldin Index (DBI)}\label{subsec:davies-bouldin-index-(dbi)}

The DBI is Defined as the Average similarity measure of each cluster with its most similar cluster, where similarity is the ration of within cluster distances
to between cluster differences.
The Goal is to minimize the DBI, as a lower DBI score indicates a better clustering division.
It has no dependency on the Shape or Density of the Cluster, It is easy to compute, and it is a Simple interpretation as it directly
quantifies the trade-off between the compactness of clusters and their separation.
The Mathematical definition for it is:
\[
    DBI = \frac{1}{k} \sum_{i=1}^k \max_{i \neq j} R(i,j)
\]
where $k$ is the number of clusters, and $R(i,j)$ is the similarity measure between clusters $i$ and $j$ where $R(i,j)$
is defined as $R(i,j) = \frac{s_i+s_i}{d_{ij}}$

$s_i$ is the average distance to all points in cluster $i$ to the centroid of cluster $i$ (intra-cluster distance)

$d_{ij}$ is the distance between the centroids of clusters $i$ and $j$ (inter-cluster distance)

$s_j$ is similarly the average intra-cluster distance for cluster $j$.

\section{Analysis of Clustering Metrics: Determining the Optimal Number of Clusters for the Al Dhannah Dataset}\label{sec:analysis-of-clustering-metrics:-determining-the-optimal-number-of-clusters-for-the-al-dhannah-dataset}

The selection of the optimal number of clusters (k) in K-means clustering is crucial for achieving the best possible grouping of data points.
We can analyze this by reviewing graphs of the Average Inertia for Different k, Average Silhouette Scores for Different k,
and a combined plot of Silhouette Score vs.Inertia so that we can find the best K for the Al Dhannah City dataset as each
dataset will have a different best K\@.

\subsection{Average Inertia for different K}\label{subsec:average-inertia-for-different-k}



As illustrated in Figure\ref{fig:1}, As we can see from the Graph, there is a sharp decline in inertia as the number of Clusters increases from 2 to around 5 followed
by a more gradual decrease.
The Lower the Inertia value, the more Compact the Clusters meaning higher quality clusters.

We can apply the elbow rule here which states that the optimal K is where the inertia curve begins to flatten in this graph the
curve begins to flatten around K=4.
this suggests that increasing the number of clusters beyond this point results in Diminishing results


\subsection{Averge Silhouette Scores for Different K}\label{subsec:averge-silhouette-scores-for-different-k}



As illustrated in Figure\ref{fig:2}, The Average Silhouette Score graph presents a different perspective, highlighting the average silhouette score of clusters as k varies.
The silhouette score measures how similar an object is to its own cluster compared to other clusters, with higher values generally indicating more appropriate clustering.

The graph shows the highest silhouette score at k=2, which then declines steadily as more clusters are added.
This indicates that at k=2, the clusters are more distinct and well-separated compared to higher values of k.

\subsection{Silhouette Score vs. Inertia}\label{subsec:silhouette-score-vs.-inertia}



As illustrated in Figure\ref{fig:3}, The combined graph Plots these scores against each other for the value of k this visual representation helps us see the
trade-off between Inertia and the Silhouette score.
If K is high in this dataset, inertia is low, so K = 2 is too much in favor of the Silhouette Score, the Clusters are separate, but the
clusters are not as dense as they should be.
K = 4, on the other hand, is perfect as The silhouette Score is not as good as K = 3 and the inertia is not as good as K = 5.

Using insights from Both previous graphs.
We can determine that K = 4 is the most balanced choice for the Current Dataset as
it provides a compromise between having distinct clusters and ensuring that the clusters are compact enough\cite{studentResults2022}.
This is suggested by the elbow in the inertia graph and relatively good silhouette scores for K = 4 in the silhouette score graph.

