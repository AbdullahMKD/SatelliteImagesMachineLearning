\thispagestyle{empty}

%TC:ignore
\section*{\centering Abstract}

This report states the process utilized in developing a python application that employs
the K-means clustering algorithm for the classification of satellite imagery.
The primary objective of this project was to simplify and enhance the usage of satellite data across various sectors
by providing a user-friendly tool for image analysis.

The application incorporates a single classifier which can be expanded to include other unsupervised classifiers as well as supervised ones
, but the K-means algorithm, which is renowned for its efficiency in segmenting images into clusters based on pixel similarity.
This method is particularly advantageous for categorizing land use and identifying patterns or changes in satellite images.

Research was conducted to ensure the optimal implementation of the K-means algorithm, with studies indicating its effectiveness
in handling large datasets and its robustness in producing significant clusters that are meaningful in the context of satellite imagery.

Designed to be intuitive, the application allows users of varying technical expertise to engage with satellite data analysis.
The system facilitates real-time processing of images, providing immediate feedback and results of the effectiveness of the Kmeans implementation,
and allows for the user to compare images overtime visually highlighting differences.
which are crucial for timely decision-making in areas such as environmental monitoring and urban planning among other areas.

%TC:endignore